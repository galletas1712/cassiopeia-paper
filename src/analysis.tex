\section{Analysis}
We assume a synchronous network model where there exists a probabilistic polynomial time adversary.
The adversary controls some committee members, which can do whatever they like.
Let \textsf{PVSS} = (\textsf{genDist}, \textsf{verifyDist}, \textsf{decrypt}, \textsf{verifyShare}, \textsf{reconstruct}) be a correct and verifiable PVSS scheme that has secrecy,
$(G, P, V)$ be a correct and extractably secure zkSNARK scheme, $(Enc, Dec)$ be a correct and secure symmetric encryption scheme,
$\mathbb{L}$ be a safe and live ledger, and $H$ be a random oracle.
We will analyze the security properties of both the non-incentivized and incentivized constructions.

\subsection{Security Analysis of Non-incentivized Construction}
\begin{theorem}{Correctness (Informal).}\label{thm:correctness_hm}
    Consider an honest dealer and a committee of size $n$ such that at least $t$ members are honest.
    The social Witness Encryption construction in Section~\ref{section:construction_without_incentives} is correct.
    See Appendix A for proof.
\end{theorem}

\begin{theorem}{Security (Informal).}\label{thm:security_hm}
    Consider an honest dealer, and a committee of size $n$ such that less than $t$ members are adversarial.
    The social Witness Encryption construction in Section~\ref{section:construction_without_incentives} is extractably secure.
    See Appendix B for proof.
\end{theorem}

At the heart of the proof of security lies the following Lemma.
Intuitively, the attack that Lemma~\ref{lemma:non_malleability} ensures protection from is a malleability attack.
In particular, an adversary may use the arguments $(c, \mathcal{R}, x, y, \pi)$ of a previous \textsf{encrypt} call to generate arguments $(c', \mathcal{R}', x', y', \pi')$ such that the decryption of $c'$ can be used to infer the decryption of $c$ (i.e. the original secret).

\begin{lemma}\label{lemma:non_malleability}
    Suppose an adversary is given the values $c, \mathcal{R}, x, y, \pi$ for secret $s$ generated by an honest dealer, and the adversary interacts with the contract.
    The adversary cannot obtain $s$ without a witness $w$ such that $(x, w) \in \mathcal{R}$. See Appendix B for proof.
\end{lemma}

Now, we analyze the security properties of the non-incentivized scheme under honest majority.
Let $h$ be the number of honest committee members.
Assume that a majority of committee members are honest (i.e. $h \geq \frac{n}{2} + 1$).
If $t = \frac{n}{2}$, then the construction is correct, since $h \geq t$.
Furthermore, the construction is also extractably secure, because the number of adversaries $n - h \leq \frac{n}{2} - 1 < t$.

Note that the construction may be instantiated with different values of $t$.
Since the minimum number of honest committee members needed to ensure correctness and security is $t$ and $n - t + 1$ respectively, higher $t$ ensures a higher degree of correctness for lower security and vice versa.
When both correctness and security are desired with the lowest possible minimum required number of honest committee members, we optimize for $\argmin_{t} \min(t, n - t + 1)$, which occurs when $t = \frac{n}{2}$, corresponding with the honest majority case.

\subsection{Security Analysis of Incentivized Construction}
We assume there exists a fixed per-block risk-free rate $r$, as outlined in Section~\ref{subsection:prelim-rfr}.
\begin{lemma}\label{lemma:honest_compensation}
    If a committee member submits a valid share within the grace period, its time-valued net compensation is at least $\beta$.
\end{lemma}
\begin{proof}[Sketch]
    Since the PVSS scheme is verifiable, the contract will only accept the dealer's PVSS ciphertext $c$ if it is valid.
    Furthermore, the contract will only accept a committee member's encrypted share if it is correct.
    Therefore, a committee member will only be paid its reward of $\frac{f}{n}$ if it submits a valid share.
    Let $d$ be the maximum lifetime of the secret.
    Since the committee member also deposits its collateral for $d$ blocks, the committee member's time-valued net compensation is
    \begin{equation*}
        \frac{f}{n} - o = \frac{f}{n} - (\frac{a}{n - t + 1} - \frac{f}{n})((1 + r)^d - 1) = \beta
        % &= \frac{f}{n} - (\frac{\frac{f}{n}(1 + r)^d - \beta}{(1 + r)^d - 1} - \frac{f}{n})((1 + r)^d - 1) \\
        % &= \beta
    \end{equation*}
\end{proof}
\begin{theorem}{Correctness (Informal).}
    Consider an honest dealer, and a committee of size $n$ such that all maintain sufficient collateral for the holding $f$ and at least $t$ members are rational.
    The social Witness Encryption construction in Section~\ref{section:construction_incentives} is correct.
\end{theorem}
\begin{proof}[Sketch]
    We follow the same proof as that of Theorem~\ref{thm:correctness_hm} but with added steps.
    If a committee member does not submit a valid share by the deadline, they forfeit both their reward and are slashed.
    By Lemma~\ref{lemma:honest_compensation}, a committee member who submits their shares will receive a net gain of $\beta$.
    Since we assume that $\beta$ is a sufficient reward, rational committee members will choose to submit valid shares of the secret on time.
    Because we assume all committee members hold sufficient collateral in the contract, $\textsf{WE.Enc}_\mathcal{R}$ completes in polynomial time.
    Since there are at least $t$ rational committee members, $S_{id}$ contains at least $t$ valid shares.
    As in the proof of Theorem~\ref{thm:correctness_hm} above, the correctness of PVSS ensures that $\textsf{reconstruct}(S_{id}) = s$, and by the correctness of the symmetric encryption scheme, $Dec_s(\hat{c}) = m$.
    Therefore, $\textsf{WE.Dec}_\mathcal{R}$ correctly recovers $m$ in polynomial time.
\end{proof}
\begin{theorem}{Payout (Informal).}
    Consider an honest dealer, and a committee of size $n$ such that all maintain sufficient collateral for the holding $f$.
    If less than $t$ committee members submit valid shares, the dealer receives $a$.
\end{theorem}
\begin{proof}[Sketch]
    Let $t'$ be the number of committee members who did not submit valid shares.
    Committee members who do not submit valid shares do not receive their reward, which is given to the dealer in Line 14 of Algorithm~\ref{alg:slash}, along with their share of the reparation fee $\frac{a}{|t'|} - \frac{f}{n}$.
    The reparation fee can always be deducted from each committee member's collateral because $\hat{b} \geq \frac{a}{|t'|} - \frac{f}{n}$.
    Therefore, the dealer receives $a$ in total.
\end{proof}