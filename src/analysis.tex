\section{Analysis}
We assume a synchronous network model where there exists a probablistic polynomial time adversary.
The adversarial controls some committee members, which can do whatever they like.
Let \textsf{PVSS} = (\textsf{genDist}, \textsf{verifyDist}, \textsf{decrypt}, \textsf{verifyShare}, \textsf{reconstruct}) be a correct and verifiable PVSS scheme that has secrecy,
$(G, P, V)$ be a correct and extractably secure zkSNARK scheme, $(Enc, Dec)$ be a correct and secure symmetric encryption scheme,
$\mathbb{L}$ be a safe and live ledger, and $H$ be a random oracle.
We will analyze the security properties of both the non-incentivized and incentivized constructions.

\subsection{Security Analysis of Non-incentivized Construction}
\begin{theorem}{Correctness}\label{thm:correctness_hm}
    Consider an honest dealer and a committee of size $n$ such that at least $t$ are honest.
    The social WE construction in Section~\ref{section:construction_without_incentives} is correct.
    See Appendix A for proof.
\end{theorem}

\begin{theorem}\label{thm:security_hm}
    Consider an honest dealer, and a committee of size $n$ such that less than $t$ are adversarial.
    The social WE construction in Section~\ref{section:construction_without_incentives} is extractably secure.
\end{theorem}
\begin{lemma}\label{lemma:non_malleability}
    Suppose an adversary is given the values $c, \mathcal{R}, x, y, \pi$ for secret $s$ generated by an honest dealer, and the adversary interacts with the contract.
    The adversary cannot obtain $s$ without a witness $w$ such that $(x, w) \in \mathcal{R}$.
\end{lemma}
\begin{proof}
    Using Lemma~\ref{lemma:non_malleability}, we can prove extractable security of the scheme.
    See Appendix B for the full proof.
\end{proof}

Now, we analyze the security properties of the non-incentivized scheme under honest majority.
Let $h$ be the number of honest committee members.
Assume that a majority of committee members are honest (i.e. $h \geq \frac{n}{2} + 1$).
If $t = \frac{n}{2}$, then the construction is correct, since $h \geq t$.
Furthermore, the construction is also extractably secure, because the number of adversaries $n - h \leq \frac{n}{2} - 1 < t$.

Note that the construction may be instantiated with different values of $t$.
Since the minimum number of honest committee members needed to ensure correctness and security is $t$ and $n - t + 1$ respectively, higher $t$ ensures a higher degree of correctness for lower security and vice versa.
When both correctness and security are desired with the lowest possible minimum required number of honest committee members, we optimize for $\argmin_{t} \min(t, n - t + 1)$, which occurs when $t = \frac{n}{2}$, corresponding with the honest majority case.
% TODO: check!

\subsection{Security Analysis of Incentivized Construction}
\begin{lemma}\label{lemma:honest_compensation}
    If a committee member submits a valid share within the grace period, their net compensation is positive.
\end{lemma}
\begin{lemma}\label{lemma:do_not_submit}
    If a committee member does not submit a valid share within the grace period, their net compensation is negative.
\end{lemma}
\begin{theorem}
    Consider an honest dealer, and a committee of size $n$ such that at least $t$ are rational.
    Assume there exists a fixed per-block risk-free rate $r$.
    The social WE construction in Section~\ref{section:construction_incentives} with respect to \textsf{PVSS}, $(G, P, V)$, $(Enc, Dec)$, $\mathbb{L}$, $H$, and $r$ is correct.
\end{theorem}
% \begin{proof}
%     % TODO: use verifiability
%     By Lemma~\ref{lemma:honest_compensation} and Lemma~\ref{lemma:do_not_submit}, a committee member is always better off if they submit a valid share of the secret within the grace period.
%     Therefore, a rational committee member will submit a valid share within the grace period.
%     Since there are at least $t$ rational committee members, $S_{id}$ contains at least $t$ valid shares.
%     As in the proof of Theorem~\ref{thm:correctness_hm} above, the correctness of PVSS ensures that $\textsf{reconstruct}(S_{id}) = s$, and by the correctness of the symmetric encryption scheme, $Dec_s(\hat{c}) = m$.
%     Since the grace period is finite, the message $m$ is decrypted correctly in finite time after $w$ has been presented, so the scheme is correct.
% \end{proof}
% TODO: different values of $t$