\paragraph{Related Work.}
General-purpose Witness Encryption was introduced by Garg et al.~\cite{witness_encryption}.
The standard security definition of Witness Encryption is \emph{extractable security}~\cite{turing_machine_fe}.
Most current WE schemes are based on multilinear maps~\cite{witness_encryption,we_multilinear_map} or on indistinguishability obfuscation (iO)~\cite{we_io}. Both construction paths require strong assumptions and are computationally impractical.
One alternative is to use weaker variants of WE that encompass only a specific subset of \textsf{NP}~\cite{MrNISC}.

% Committee
Another alternative to build practical WE, which we explore in this work, is to do \emph{social witness encryption} where a committee is entrusted up to an adversarial threshold.
Goyal et al.~\cite{eweb} were the first to propose an extractable social WE scheme constructed from Proactive Secret Sharing (PSS) that leverages a blockchain.
The construction can use a Proof-of-Stake blockchain's validators as the committee.
It also allows for the witness to remain private after decryption.
One particularly useful application of WE and blockchains is TE~\cite{timelock_puzzles,timelock_from_crc,timed_release_cryptography}.
Social TE leveraging Identity-Based Encryption (IBE)~\cite{ibe} has been put forth in i-TiRE~\cite{i-TiRE} and tlock~\cite{tlock}.
Dottling et al.~\cite{mcfly} proposed a social TE scheme using a weaker Signature-based Witness Encryption (SWE) scheme.

The critical building block for social TE and WE is a Secret Sharing Scheme~\cite{shamir_ss}. In the blockchain setting, threshold encryption was previously explored in works by Benhamouda et al.~\cite{benhamouda_ecpss} and Goyal et al.~\cite{eweb}.
For accountability, we use Publicly Verifiable Secret Sharing (PVSS)~\cite{first_pvss_chor,pvss_stadler}.
For our implementation, we use SCRAPE~\cite{pvss_scrape} as the underlying PVSS scheme.

% TODO: In order to have a shifting committee, a Proactive Secret Sharing Scheme (PSS) is required. If accountability is required, then Publicly Verifiable Secret Sharing (PVSS) is required. In this work, we use a PVSS and our committee is static.
% What to say here?
% PVSS?

% Homomorphic Encryption Random Becaon, Scalable Bias-Resistant Randomness - more randomness beacons
% Collusion-deterrent information escrow - OT and Watermarking + incentives
%
% i-TiRE - timelock encryption with threshold IBE with forward compatibility of decryption keys

% Timelock encryption - timelock puzzles and trusted servers
% Timelock puzzles are expensive and inaccurate
% Trusted servers approach is usually combined with IBE
% Newer works relax trust assumption with DKG
% Witness encryption - impractical in standard model. "Differing inputs" obfuscation?
% Secret sharing - Shamir, VSS, PVSS (dynamic?)

% Constructions of timelock and witness encryption that leverage properties of blockchains have been proposed.
% For instance, Liu et al.~\cite{timelock_from_crc} proposed a construction that leverages large public computations such as Bitcoin mining and extractable witness encryption to perform timelock encryption without relying on trusted third parties nor producing significant exogenous computational overhead.

% One of the primary underlying assumptions in all blockchains is that the committees that secure blockchains have an honest majority.
% Given such a committee, it is therefore practical to construct previously impractical cryptographic protocols based off of threshold secret sharing, which relies on the same honest majority assumption.
% For instance, \emph{drand} \cite{drand} is a fixed-committee randomness beacon produced by generating BLS threshold signatures \cite{bls} in rounds, where timelock encryption is derived by leveraging identity-based encryption \cite{ibe} using the round number as the public key and the threshold signature as a secret key.
% Although drand is highly efficient for timelock encryption, the use of identity-based encryption does not allow it be used for extractable witness encryption, as round numbers cannot be used as public keys.
% Goyal et al. \cite{eweb} presents a definition of witness encryption security in the blockchain setting. Furthermore, the authors also propose an extractable witness encryption construction (and by extension timelock encryption) using an optimized dynamic proactive secret sharing scheme, where the secret is reshared with each committee at each time point.
% However, due to the requirement of resharing for the dynamic committee, it is not practical to implement in a smart contract as the gas cost of resharing can be unpredictably high, especially when the secret is kept for a long time.
% Any more of these kinds of papers?
% Do we need to cite DPSS, VSS, PVSS papers?