\section{Introduction}

% Hook - a will, a secret encrypted into the future - literary (message in a bottle)
% Problem statement (people wanted to do timelock and witness encryption for a long time)
% Without blockchains it's completely impractical
% New assumption of honest majority in blockchains - ask a committee to keep a secret and release it for us after a certain time
% We want to make this composable for use in DeFi
% Make accessible as a service on chain - accountability
% In the blockchain setting, timelock is only a special case of Witness Encryption

% TODO: hook
Witness Encryption (WE)~\cite{witness_encryption} is a cryptographic scheme where a message is encrypted such that it can only be decrypted when a solution (also called the ``witness'') to a computational puzzle is presented.
For example, a message may be witness-encrypted such that it can only be decrypted if for some connected graph, a Hamiltonian cycle is found.
A particularly useful instantiation of WE in the blockchain setting is Timelock Encryption (TE)~\cite{timelock_puzzles,timelock_from_crc,timed_release_cryptography}, where a message is encrypted such that it can only be decrypted after a set unlock time.
This enables important use cases in DeFi such as sealed bid auctions and front-running prevention~\cite{i-TiRE}.

Though WE is still impractical under standard assumptions, the ``honest majority'' assumption in modern blockchains can be leveraged to make WE practical.
We do so by asking a committee to keep the secret and release it for us when the witness is presented.
As long as a majority of the committee is honest, the secret is revealed at the right time.
Our goal is to make WE practical and composable for use in DeFi.
Furthermore, we hold committee members accountable for releasing correct decryption keys in a timely manner.

\import{./introduction}{contributions.tex}
\import{./introduction}{construction_overview.tex}
\import{./introduction}{related_work.tex}
\import{./introduction}{applications.tex}