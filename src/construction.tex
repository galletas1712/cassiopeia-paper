\section{Construction}\label{section:construction_without_incentives}
The Cassiopeia smart contract is instantiated for a fixed committee with known public keys of size $n$ and a threshold $t$.
The threshold $t$ is a public parameter indicating the minimum number of honest committee members required for the construction to be secure.

Suppose the dealer wants to perform \emph{social} Witness Encryption with public witness security on a message $m$ (Algorithm~\ref{alg:dealer_no_incentives}).
First, the dealer chooses a relation $\mathcal{R}$ and a corresponding instance $x$.
% In practice, the pair $(\mathcal{R}, x)$ could be encoded as a smart contract function that accepts $w$ and returns 1 if $(x, w) \in \mathcal{R}$ and 0 otherwise.
Then, the dealer generates a random bit string $s$ that can be simultaneously used as the secret in a PVSS scheme and the key of a symmetric encryption scheme. % Security of symmetric encryption scheme?
The dealer runs \textsf{PVSS.genDist}$(s, [pk_i])$ to generate the encrypted secret shares $[\hat{s}_i]$ along with a proof $\pi_D$ that the generated encrypted secret shares are consistent with one another.
Note that a valid proof $\pi_D$ guarantees that \textsf{Cassiopeia.encrypt} prevents committee members from maintaining shares of invalid secrets.
It becomes vital to security in the incentivized construction in Section~\ref{section:construction_incentives}.

Denote $c = ([\hat{s}_i], \pi_D)$ as the \emph{PVSS ciphertext}.
Using a symmetric encryption scheme, the dealer encrypts $m$ with the key $s$ to produce $\hat{c} = Enc_s(m)$.
The dealer calls the smart contract function \textsf{encrypt} to register the PVSS ciphertext and instance on chain.
The contract checks whether the proof $\pi_D$ is valid, and if so, makes the encrypted secret shares available to the committee members. 
A unique identifier for the secret $id$ is returned.
Subsequently, $\hat{c}$ is dispersed publicly, either off-chain to optimize gas costs or on-chain for data availability.

\begin{algorithm}[h]
\caption{Subroutine: dealer performing Witness Encryption}
\label{alg:dealer_no_incentives}
    \begin{algorithmic}[1]
        \Function{$\text{\sf WE.Enc}_{\mathcal{R}}$}{$1^{\lambda}, x, m$}
            \State $s \getsrandomly \{0,1\}^{\lambda}$
            \State $\hat{c} \gets Enc_s(m)$
            \State $c \gets \textsf{PVSS.genDist}(s, [pk_i])$
            \State $y \gets H(s \concat c \concat \mathcal{R} \concat x)$
            \State $\pi \gets P(\sigma, (c, \mathcal{R}, x, y, [pk_i]), s)$
            \State $id \gets \textsf{Cassiopeia.encrypt}(c, \mathcal{R}, x, y, \pi)$
            \State $\textsf{disperse}(id, \hat{c})$ % IMPORTANT: dealer MUST be honest for secret recovery right?
        \EndFunction
    \end{algorithmic}
\end{algorithm}

Anybody who obtains a valid witness $w$ can call the smart contract function \textsf{claim} to start decryption (Algorithm~\ref{alg:decryption_no_incentives}).
The smart contract checks that $w$ is indeed a valid witness such that $(x, w) \in \mathcal{R}$.
If so, a flag $M_{id}$ is set indicating the secret has been claimed.
Note that to support private decryption, one could provide a zkSNARK proof as the witness $w$ for a boolean circuit corresponding to $x$ and $\mathcal{R}$~\cite{eweb}.
This way, $w$ reveals nothing about the actual secret witness while still retaining the same correctness and security properties.

Now, committee members will decrypt their encrypted shares $\hat{s}_i$ and submit the result on chain (Algorithm~\ref{alg:committee_member_no_incentives}).
A committee member does so by first using \textsf{PVSS.decrypt} to obtain a decryption $s'_i$ and a proof $\pi_i$ that $s'_i$ is a valid decryption, i.e. $s'_i = s_i$.
The committee member then submits the share on-chain by calling the smart contract function \textsf{submitShare}, which verifies the proof and stores $s'_i$ in the set of decrypted shares $S_{id}$ inside the contract.
Once $|S_{id}| \geq t$, anyone can reconstruct the secret $s$ using \textsf{PVSS.reconstruct}.
To obtain the original message, $\hat{c}$ is fetched from public storage and decrypted using $s$ as the key to produce $m = Dec_s(\hat{c})$.
Note that for our correctness proof to follow, we assume committee members submit their shares within $\Delta$ blocks of the secret being claimed, where $\Delta$ is a fixed parameter.
The full decryption procedure is written in pseudocode below.

\begin{algorithm}[h]
    \caption{Decryption Procedure (performed by anybody)}
    \label{alg:decryption_no_incentives}
    \begin{algorithmic}[1]
        \Let{id}{\text{identifier of secret to decrypt}}
        \Function{$\text{\sf WE.Dec}_{\mathcal{R}}$}{$1^{\lambda}, c, x, w$}
            \State $\textsf{Cassiopeia.claim}(id, w)$
            \CommentLine{Wait for committee members to submit shares}
            \On{$|\textsf{Cassiopeia}.S_{id}| \geq t$}
                \State $s \gets \textsf{reconstruct}(\textsf{Cassiopeia}.S_{id})$
                \State $\textsf{fetch}(\hat{c})$
                \State $m \gets Dec_s(\hat{c})$
            \EndOn
        \EndFunction
    \end{algorithmic}
\end{algorithm}

\begin{algorithm}[h]
    \caption{Subroutine: committee members submitting shares}
\label{alg:committee_member_no_incentives}
    \begin{algorithmic}[1]
        \Let{i}{\text{index of own public key in $[pk_i]$}}
        \On{$\textsf{Cassiopeia}.M_{id} = \textsc{claimed}$}
            \State $([\hat{s}_i], \pi_D) \gets \textsf{Cassiopeia}.C_{id}.c$
            \State $(s'_i, \pi_i) \gets \textsf{PVSS.decrypt}(\hat{s}_i, sk_i)$
            \State $\textsf{Cassiopeia.submitShare}(s'_i, \pi_i, id, i)$
        \EndOn
    \end{algorithmic}
\end{algorithm}

However, this protocol still vulnerable to malleability attacks.
Concretely, let $x'$ be an instance $x'$ of possibly another relation $\mathcal{R}'$ for which the adversary already knows a valid witness $w'$.
The adversary can act as a malicious dealer, calling \textsf{encrypt} with $c$ and $x'$ instead of $x$, then call \textsf{claim} with $w'$ to notify committee members to start submitting their shares (and thereby start decryption).
Therefore, the adversary can bypass the requirement of finding a valid witness $w$ for $x$ to decrypt the secret encoded by $c$.

To mitigate this issue, we ask the dealer to provide a proof in zero knowledge that he knows $s$ and he intends to encrypt $s$ with the instance $x$.
In particular, the dealer generates a commitment to the secret and ciphertext, tying it to $\mathcal{R}$ and $x$ by computing $y = H(s \concat c \concat \mathcal{R} \concat x)$, where $H$ is a hash function that approximates a random oracle.
The dealer then generates a zkSNARK proof of knowledge of $s$ (Algorithm~\ref{alg:snark_circuit}) such that if the encrypted shares were decrypted and recombined, the result would be $s$, and that $y = H(s \concat c \concat \mathcal{R} \concat x)$.
More formally, the proof $\pi = P(\sigma, (c, \mathcal{R}, x, y, [pk_i]), s)$.
Without knowledge of $s$, an adversarial dealer cannot use the same ciphertext $c$ with another instance $x'$.

\begin{algorithm}
    \caption{zkSNARK circuit defined by $\mathcal{R}_C$}
    \label{alg:snark_circuit}
    \begin{algorithmic}[1]
        \Require $x_C = (c, \mathcal{R}, x, y, [pk_i])$, $w_C = s$
        \State $y' \gets H(s \concat c \concat \mathcal{R}, x)$
        \State $c' \gets \textsf{PVSS.genDist}(s, [pk_i])$
        \State \Return $y' = y \land c' = c$
    \end{algorithmic}
\end{algorithm}

When the dealer calls \textsf{encrypt}, he must include $y$ and $\pi$.
The contract verifies that the zero knowledge proof $\pi$ is valid with respect to $c$, $\mathcal{R}$ and $y$, on top of already verifying the PVSS ciphertext as outlined above.
Concretely, the $\pi$ must be valid according to $V(\tau, (c, \mathcal{R}, x, y, [pk_i]), \pi) = 1$.
Without a valid $\pi$, an adversary would not be able to carry out a malleability attack.
The Cassiopeia smart contract is written in pseudocode in Algorithm~\ref{alg:cassiopeia_no_incentives}.

\begin{algorithm}[h]
\caption{Cassiopeia Smart Contract}
\label{alg:cassiopeia_no_incentives}
    \begin{algorithmic}[1]
        \Contract{Cassiopeia}
            \Let{C, S, M}{\emptyset}
            
            \Function{\sf encrypt}{$c, \mathcal{R}, x, y, \pi$}
                \State $([\hat{s}_i], \pi_D) \gets c$
                \State require($V(\tau, (c, \mathcal{R}, x, y, [pk_i]), \pi) = 1 \land \textsf{PVSS.verifyDist}([\hat{s}_i], [pk_i], \pi_D$)
                \State $id \gets H(c, \mathcal{R}, x)$
                \State $C_{id} \gets (c, \mathcal{R}, x)$
                \State \Return $id$
            \EndFunction
            
            \Function{\sf claim}{$id, w$}
                \State $\{\mathcal{R}, x, \dots\} \gets C_{id}$
                \State require($(x, w) \in \mathcal{R}$)
                \State $M_{id} \gets \textsc{claimed}$
            \EndFunction
            
            \Function{\sf submitShare}{$s_i', \pi_i, id, i$}
                \State require($M_{id} = \textsc{claimed}$)
                \State $([\hat{s}_i], \bot) \gets C_{id}.c$
                \State $\textsf{PVSS.verifyShare}(i, \hat{s}_i, s_i', \pi_i)$
                \State $S_{id,i} \gets s$
            \EndFunction
        \EndContract
    \end{algorithmic}
\end{algorithm}
