\section{Introduction}

% Hook - a will, a secret encrypted into the future - literary (message in a bottle)
% Problem statement (people wanted to do timelock and witness encryption for a long time)
% Without blockchains it's completely impractical
% New assumption of honest majority in blockchains - ask a committee to keep a secret and release it for us after a certain time
% We want to make this composable for use in DeFi
% Make accessible as a service on chain - accountability
% In the blockchain setting, timelock is only a special case of Witness Encryption

% TODO: hook
Witness Encryption ~\cite{witness_encryption} is a cryptographic scheme where a message is encrypted such that it can only be decrypted when a solution (also called the ``witness'') to a computational puzzle is presented.
For example, a message may be witness-encrypted such that it can only be decrypted if for some connected graph, a Hamiltonian cycle is found.
A particularly useful instantiation of Witness Encryption in the blockchain setting is Timelock Encryption ~\cite{timelock_puzzles,timelock_from_crc,timed_release_cryptography}, where a message is encrypted such that it can only be decrypted after a set unlock time.
This enables important use cases in DeFi such as sealed bid auctions and front-running prevention~\cite{i-TiRE}.

Though Witness Encryption is still impractical under standard assumptions, the ``honest majority'' assumption in modern blockchains can be leveraged to make Witness Encryption practical.
We do so by asking a committee to keep the secret and release it for us when the witness is presented.
As long as a majority of the committee is honest, the secret is revealed at the right time.
Our goal is to make Witness Encryption practical and composable for use in DeFi.
Furthermore, we hold committee members accountable for releasing correct decryption keys in a timely manner.

\paragraph{Our Contributions.} We summarize our contributions as follows:
\begin{enumerate}
    \item We put forth a smart contract-based construction for Witness Encryption.
    \item We explore whether our scheme is correct and secure under the honest majority and rational majority settings.
    \item We implement the construction as an open-source Solidity smart contract, benchmark gas usage on Ethereum, and provide our parameterization of the scheme.
\end{enumerate}

Traditionally Witness Encryption schemes require the witness to remain private during decryption.
However, in many practical use cases, a public witness is useful as the secret does not remain hidden.
We focus on constructing Witness Encryption with public witness security, but it is possible to extend our scheme to support private decryption.

\paragraph{Construction Overview.}
Suppose Alice wants to encrypt a secret with an instance of a puzzle.
Alice will entrust a fixed committee to keep her secret and release it if someone presents a solution to the puzzle to the smart contract.
She trusts that at a majority of the committee are honest.
Alice splits her secret into shares, one for each committee member, such that a majority of them are required to reconstruct the original secret and each committee member is only able to decrypt its own share.
Alice then calls a smart contract function to publish the encrypted shares for them to access.
She also attaches a zero-knowledge proof of knowledge of the secret to ensure she is honest and actually knows the secret.
The contract verifies that Alice split her secret into shares correctly.
When the contract receives a solution to the puzzle, it enables committee members to submit their decrypted shares to the contract.
When a committee member submits a share, the contract verifies that it is consistent with the corresponding encrypted share Alice posted before accepting the share.
As long as at least a majority of the committee members revealed their shares, Alice's secret can be reconstructed by anyone.

We also propose an incentivized version of the above scheme which includes fees and slashing.
Each committee member deposits a stake in the contract.
If a committee member correctly reveals its decrypted shares, it is awarded a fee.
Otherwise, its stake is slashed.
We ensure the fee is sufficient to incentivize honest committee member participation, using the risk-free rate as a benchmark.
Note that the incentivized version of the scheme requires Alice to provide a deadline upfront at which the secret can be decrypted even without a witness.

\paragraph{Related Work.}
General-purpose Witness Encryption was introduced by Garg et al.~\cite{witness_encryption}.
The standard security definition of Witness Encryption is \emph{extractable security}~\cite{turing_machine_fe}.
Most current Witness Encryption schemes are based on multilinear maps~\cite{witness_encryption,we_multilinear_map} or on indistinguishability obfuscation (iO)~\cite{we_io}. Both construction paths require strong assumptions and are computationally impractical.
One alternative is to use weaker variants of Witness Encryption that encompass only a specific subset of \textsf{NP}~\cite{MrNISC}.

% Committee
Another alternative to build practical Witness Encryption, which we explore in this work, is to do \emph{social witness encryption} where a committee is entrusted up to an adversarial threshold.
Goyal et al.~\cite{eweb} were the first to propose an extractable social Witness Encryption scheme constructed from Proactive Secret Sharing (PSS) that leverages a blockchain.
The construction can use a Proof-of-Stake blockchain's validators as the committee.
It also allows for the witness to remain private after decryption.
One particularly useful application of Witness Encryption and blockchains is Timelock Encryption~\cite{timelock_puzzles,timelock_from_crc,timed_release_cryptography}.
Social Timelock Encryption leveraging Identity-Based Encryption (IBE)~\cite{ibe} has been put forth in i-TiRE~\cite{i-TiRE} and tlock~\cite{tlock}.
Dottling et al.~\cite{mcfly} proposed a social Timelock Encryption scheme using a weaker Signature-based Witness Encryption (SWE) scheme.

The critical building block for social Timelock Encryption and Witness Encryption is a Secret Sharing Scheme~\cite{shamir_ss}. In the blockchain setting, threshold encryption was previously explored in works by Benhamouda et al.~\cite{benhamouda_ecpss} and Goyal et al.~\cite{eweb}.
For accountability, we use Publicly Verifiable Secret Sharing (PVSS)~\cite{first_pvss_chor,pvss_stadler}.
For our implementation, we use SCRAPE~\cite{pvss_scrape} as the underlying PVSS scheme.

\paragraph{Applications.}
Witness Encryption can be used to build powerful cryptographic primitives such as succinct Functional Encryption~\cite{functional_encryption} for Turing machines~\cite{turing_machine_fe}.
When combined with a blockchain, Witness Encryption can also be leveraged to achieve fairness against dishonest majority in a secure multi-party computation protocol~\cite{we_mpc_fairness}.