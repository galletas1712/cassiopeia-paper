\section{Preliminaries}
    \import{./preliminaries}{secret_sharing.tex}
    \import{./preliminaries}{pvss.tex}
    \import{./preliminaries}{zk_snark.tex}
    \import{./preliminaries}{witness_encryption.tex}
    \subsection{Risk-Free Rate}
    We assume there exists a fixed per-block risk-free rate $r$ that is agreed upon by everyone.
    At any point in time, some amount of money $M$ can be deposited in some external source to yield $M(1 + r)$ after one block.
    If $M$ is deposited for $k$ blocks, then $M(1 + r)^k$.
    Therefore, if $M$ is locked in escrow, the owner of $M$ incurs an opportunity cost of $M((1 + r)^k - 1)$.

	% \subsection{Timelock Encryption} % TODO: write this up
 %    In the standard model, TE is implemented either through time-lock puzzles~\cite{timed_release_cryptography,timelock_puzzles}, which are computationally-heavy puzzles that cannot be completed without running for a certain amount of time.
 %    However, blockchains provide a common reference clock that can be leveraged as a source of truth for how much time has passed, and can be combined with extractable WE to build TE~\cite{timelock_from_crc}.
 %    An efficient extractable WE scheme would therefore allow TE (and Functional Encryption as well) to be practical.
