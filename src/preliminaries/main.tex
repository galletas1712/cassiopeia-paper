\section{Preliminaries}
    \subsection{Symmetric Encryption}
    Let $s$ be the secret key and $m$ be the message to encrypt.
    A symmetric encryption scheme contains an encryption algorithm $Enc_s(m)$ that produces a ciphertext $c$ and a decryption algorithm $Dec_s(c)$ that produces the original message $m$.
    \import{./preliminaries}{secret_sharing.tex}
    \import{./preliminaries}{pvss.tex}
    \import{./preliminaries}{zk_snark.tex}
    \import{./preliminaries}{witness_encryption.tex}
    \subsection{Risk-Free Rate}\label{subsection:prelim-rfr}
    Our protocol assumes there exists a per-block risk-free rate $r$ agreed upon by all parties participating in the protocol.
    At any block, anyone can deposit $M$ and earn $M(1 + r)$ after one block via an exogenous risk-free source of yield.
    The oppportunity cost for not earning risk-free yield on $M$ for $k$ blocks is $M(1 + r)^k - M$.

	% \subsection{Timelock Encryption} % TODO: write this up
 %    In the standard model, TE is implemented either through time-lock puzzles~\cite{timed_release_cryptography,timelock_puzzles}, which are computationally-heavy puzzles that cannot be completed without running for a certain amount of time.
 %    However, blockchains provide a common reference clock that can be leveraged as a source of truth for how much time has passed, and can be combined with extractable WE to build TE~\cite{timelock_from_crc}.
 %    An efficient extractable WE scheme would therefore allow TE (and Functional Encryption as well) to be practical.
